% Options for packages loaded elsewhere
\PassOptionsToPackage{unicode}{hyperref}
\PassOptionsToPackage{hyphens}{url}
%
\documentclass[
]{article}
\usepackage{amsmath,amssymb}
\usepackage{lmodern}
\usepackage{ifxetex,ifluatex}
\ifnum 0\ifxetex 1\fi\ifluatex 1\fi=0 % if pdftex
  \usepackage[T1]{fontenc}
  \usepackage[utf8]{inputenc}
  \usepackage{textcomp} % provide euro and other symbols
\else % if luatex or xetex
  \usepackage{unicode-math}
  \defaultfontfeatures{Scale=MatchLowercase}
  \defaultfontfeatures[\rmfamily]{Ligatures=TeX,Scale=1}
\fi
% Use upquote if available, for straight quotes in verbatim environments
\IfFileExists{upquote.sty}{\usepackage{upquote}}{}
\IfFileExists{microtype.sty}{% use microtype if available
  \usepackage[]{microtype}
  \UseMicrotypeSet[protrusion]{basicmath} % disable protrusion for tt fonts
}{}
\makeatletter
\@ifundefined{KOMAClassName}{% if non-KOMA class
  \IfFileExists{parskip.sty}{%
    \usepackage{parskip}
  }{% else
    \setlength{\parindent}{0pt}
    \setlength{\parskip}{6pt plus 2pt minus 1pt}}
}{% if KOMA class
  \KOMAoptions{parskip=half}}
\makeatother
\usepackage{xcolor}
\IfFileExists{xurl.sty}{\usepackage{xurl}}{} % add URL line breaks if available
\IfFileExists{bookmark.sty}{\usepackage{bookmark}}{\usepackage{hyperref}}
\hypersetup{
  pdftitle={HW2},
  pdfauthor={MGinorio},
  hidelinks,
  pdfcreator={LaTeX via pandoc}}
\urlstyle{same} % disable monospaced font for URLs
\usepackage[margin=1in]{geometry}
\usepackage{color}
\usepackage{fancyvrb}
\newcommand{\VerbBar}{|}
\newcommand{\VERB}{\Verb[commandchars=\\\{\}]}
\DefineVerbatimEnvironment{Highlighting}{Verbatim}{commandchars=\\\{\}}
% Add ',fontsize=\small' for more characters per line
\usepackage{framed}
\definecolor{shadecolor}{RGB}{248,248,248}
\newenvironment{Shaded}{\begin{snugshade}}{\end{snugshade}}
\newcommand{\AlertTok}[1]{\textcolor[rgb]{0.94,0.16,0.16}{#1}}
\newcommand{\AnnotationTok}[1]{\textcolor[rgb]{0.56,0.35,0.01}{\textbf{\textit{#1}}}}
\newcommand{\AttributeTok}[1]{\textcolor[rgb]{0.77,0.63,0.00}{#1}}
\newcommand{\BaseNTok}[1]{\textcolor[rgb]{0.00,0.00,0.81}{#1}}
\newcommand{\BuiltInTok}[1]{#1}
\newcommand{\CharTok}[1]{\textcolor[rgb]{0.31,0.60,0.02}{#1}}
\newcommand{\CommentTok}[1]{\textcolor[rgb]{0.56,0.35,0.01}{\textit{#1}}}
\newcommand{\CommentVarTok}[1]{\textcolor[rgb]{0.56,0.35,0.01}{\textbf{\textit{#1}}}}
\newcommand{\ConstantTok}[1]{\textcolor[rgb]{0.00,0.00,0.00}{#1}}
\newcommand{\ControlFlowTok}[1]{\textcolor[rgb]{0.13,0.29,0.53}{\textbf{#1}}}
\newcommand{\DataTypeTok}[1]{\textcolor[rgb]{0.13,0.29,0.53}{#1}}
\newcommand{\DecValTok}[1]{\textcolor[rgb]{0.00,0.00,0.81}{#1}}
\newcommand{\DocumentationTok}[1]{\textcolor[rgb]{0.56,0.35,0.01}{\textbf{\textit{#1}}}}
\newcommand{\ErrorTok}[1]{\textcolor[rgb]{0.64,0.00,0.00}{\textbf{#1}}}
\newcommand{\ExtensionTok}[1]{#1}
\newcommand{\FloatTok}[1]{\textcolor[rgb]{0.00,0.00,0.81}{#1}}
\newcommand{\FunctionTok}[1]{\textcolor[rgb]{0.00,0.00,0.00}{#1}}
\newcommand{\ImportTok}[1]{#1}
\newcommand{\InformationTok}[1]{\textcolor[rgb]{0.56,0.35,0.01}{\textbf{\textit{#1}}}}
\newcommand{\KeywordTok}[1]{\textcolor[rgb]{0.13,0.29,0.53}{\textbf{#1}}}
\newcommand{\NormalTok}[1]{#1}
\newcommand{\OperatorTok}[1]{\textcolor[rgb]{0.81,0.36,0.00}{\textbf{#1}}}
\newcommand{\OtherTok}[1]{\textcolor[rgb]{0.56,0.35,0.01}{#1}}
\newcommand{\PreprocessorTok}[1]{\textcolor[rgb]{0.56,0.35,0.01}{\textit{#1}}}
\newcommand{\RegionMarkerTok}[1]{#1}
\newcommand{\SpecialCharTok}[1]{\textcolor[rgb]{0.00,0.00,0.00}{#1}}
\newcommand{\SpecialStringTok}[1]{\textcolor[rgb]{0.31,0.60,0.02}{#1}}
\newcommand{\StringTok}[1]{\textcolor[rgb]{0.31,0.60,0.02}{#1}}
\newcommand{\VariableTok}[1]{\textcolor[rgb]{0.00,0.00,0.00}{#1}}
\newcommand{\VerbatimStringTok}[1]{\textcolor[rgb]{0.31,0.60,0.02}{#1}}
\newcommand{\WarningTok}[1]{\textcolor[rgb]{0.56,0.35,0.01}{\textbf{\textit{#1}}}}
\usepackage{longtable,booktabs,array}
\usepackage{calc} % for calculating minipage widths
% Correct order of tables after \paragraph or \subparagraph
\usepackage{etoolbox}
\makeatletter
\patchcmd\longtable{\par}{\if@noskipsec\mbox{}\fi\par}{}{}
\makeatother
% Allow footnotes in longtable head/foot
\IfFileExists{footnotehyper.sty}{\usepackage{footnotehyper}}{\usepackage{footnote}}
\makesavenoteenv{longtable}
\usepackage{graphicx}
\makeatletter
\def\maxwidth{\ifdim\Gin@nat@width>\linewidth\linewidth\else\Gin@nat@width\fi}
\def\maxheight{\ifdim\Gin@nat@height>\textheight\textheight\else\Gin@nat@height\fi}
\makeatother
% Scale images if necessary, so that they will not overflow the page
% margins by default, and it is still possible to overwrite the defaults
% using explicit options in \includegraphics[width, height, ...]{}
\setkeys{Gin}{width=\maxwidth,height=\maxheight,keepaspectratio}
% Set default figure placement to htbp
\makeatletter
\def\fps@figure{htbp}
\makeatother
\setlength{\emergencystretch}{3em} % prevent overfull lines
\providecommand{\tightlist}{%
  \setlength{\itemsep}{0pt}\setlength{\parskip}{0pt}}
\setcounter{secnumdepth}{-\maxdimen} % remove section numbering
\ifluatex
  \usepackage{selnolig}  % disable illegal ligatures
\fi

\title{HW2}
\author{MGinorio}
\date{2/12/2022}

\begin{document}
\maketitle

\hypertarget{ch-2}{%
\subsection{CH 2}\label{ch-2}}

\hypertarget{amarr}{%
\subsection{AMARR}\label{amarr}}

\hypertarget{section}{%
\paragraph{2.1}\label{section}}

The web site www.playbill.com provides weekly reports on the box office
ticket sales for plays on Broadway in New York. We shall consider the
data for the week October 11--17, 2004 (referred to below as the current
week). The data are in the form of the gross box office results for the
current week and the gross box office results for the previous week
(i.e., October 3--10, 2004). The data, plotted in Figure 2.6 , are
available on the book web site in the file playbill.csv. Fit the
following model to the data:
\[ Y =  \beta_{0}+ \beta_1 \overline{x} + e\]where Y is the gross box
office results for the current week (in \$) and x is the gross box
office results for the previous week (in \$).

Complete the following tasks:

\begin{Shaded}
\begin{Highlighting}[]
\FunctionTok{library}\NormalTok{(readr)}
\FunctionTok{library}\NormalTok{(tidyverse)}
\FunctionTok{library}\NormalTok{(skimr)}

\CommentTok{\#load data}
\NormalTok{playbill }\OtherTok{\textless{}{-}} \FunctionTok{read\_csv}\NormalTok{(}\StringTok{"https://raw.githubusercontent.com/mgino11/Business\_Analytics/main/621{-}Homeworks/playbill.csv"}\NormalTok{)}

\NormalTok{playbill }\SpecialCharTok{\%\textgreater{}\%} \FunctionTok{skim}\NormalTok{()}
\end{Highlighting}
\end{Shaded}

\begin{longtable}[]{@{}ll@{}}
\caption{Data summary}\tabularnewline
\toprule
& \\
\midrule
\endfirsthead
\toprule
& \\
\midrule
\endhead
Name & Piped data \\
Number of rows & 18 \\
Number of columns & 3 \\
\_\_\_\_\_\_\_\_\_\_\_\_\_\_\_\_\_\_\_\_\_\_\_ & \\
Column type frequency: & \\
character & 1 \\
numeric & 2 \\
\_\_\_\_\_\_\_\_\_\_\_\_\_\_\_\_\_\_\_\_\_\_\_\_ & \\
Group variables & None \\
\bottomrule
\end{longtable}

\textbf{Variable type: character}

\begin{longtable}[]{@{}lrrrrrrr@{}}
\toprule
skim\_variable & n\_missing & complete\_rate & min & max & empty &
n\_unique & whitespace \\
\midrule
\endhead
Production & 0 & 1 & 4 & 24 & 0 & 18 & 0 \\
\bottomrule
\end{longtable}

\textbf{Variable type: numeric}

\begin{longtable}[]{@{}lrrrrrrrrrl@{}}
\toprule
skim\_variable & n\_missing & complete\_rate & mean & sd & p0 & p25 &
p50 & p75 & p100 & hist \\
\midrule
\endhead
CurrentWeek & 0 & 1 & 617842.8 & 297812.9 & 105853 & 484958.0 & 586800.0
& 788322.8 & 1180266 & ▃▅▇▃▂ \\
LastWeek & 0 & 1 & 622186.6 & 302724.5 & 105698 & 491210.2 & 590876.5 &
801578.5 & 1202536 & ▃▅▇▃▂ \\
\bottomrule
\end{longtable}

\begin{Shaded}
\begin{Highlighting}[]
\CommentTok{\#stablish relationship using scatter plot}

\NormalTok{playbill }\SpecialCharTok{\%\textgreater{}\%} 
  \FunctionTok{ggplot}\NormalTok{(}\FunctionTok{aes}\NormalTok{(}\AttributeTok{x=}\NormalTok{LastWeek, }\AttributeTok{y=}\NormalTok{CurrentWeek)) }\SpecialCharTok{+} 
  \FunctionTok{geom\_point}\NormalTok{() }\SpecialCharTok{+}
  \FunctionTok{labs}\NormalTok{(}\AttributeTok{title =} \StringTok{"Relationship of gross Results btw Last and Current Week"}\NormalTok{) }\SpecialCharTok{+} 
  \FunctionTok{geom\_smooth}\NormalTok{(}\AttributeTok{method =} \StringTok{"lm"}\NormalTok{, }\AttributeTok{se =} \ConstantTok{FALSE}\NormalTok{)}
\end{Highlighting}
\end{Shaded}

\begin{verbatim}
## `geom_smooth()` using formula 'y ~ x'
\end{verbatim}

\includegraphics{HW2_files/figure-latex/unnamed-chunk-1-1.pdf}

\begin{Shaded}
\begin{Highlighting}[]
\CommentTok{\# The se = FALSE argument suppresses standard error uncertainty bars}
\end{Highlighting}
\end{Shaded}

\hypertarget{eda}{%
\paragraph{EDA}\label{eda}}

\begin{itemize}
\item
  CurrentWeek \(y\) (gross box office results for the current week)
\item
  LastWeek \(x\) (gross box office results for the previous week)
\end{itemize}

\begin{Shaded}
\begin{Highlighting}[]
\FunctionTok{library}\NormalTok{(moderndive)}

\NormalTok{playbill }\SpecialCharTok{\%\textgreater{}\%} 
  \FunctionTok{get\_correlation}\NormalTok{(}\AttributeTok{formula =}\NormalTok{ CurrentWeek}\SpecialCharTok{\textasciitilde{}}\NormalTok{LastWeek)}
\end{Highlighting}
\end{Shaded}

\begin{verbatim}
## # A tibble: 1 x 1
##     cor
##   <dbl>
## 1 0.998
\end{verbatim}

\begin{Shaded}
\begin{Highlighting}[]
\CommentTok{\# fit regression model }
\NormalTok{pbill\_model }\OtherTok{\textless{}{-}} \FunctionTok{lm}\NormalTok{(CurrentWeek}\SpecialCharTok{\textasciitilde{}}\NormalTok{LastWeek, }\AttributeTok{data =}\NormalTok{ playbill )}

\CommentTok{\#Get REgression Table}
\FunctionTok{get\_regression\_table}\NormalTok{(pbill\_model)}
\end{Highlighting}
\end{Shaded}

\begin{verbatim}
## # A tibble: 2 x 7
##   term      estimate std_error statistic p_value   lower_ci upper_ci
##   <chr>        <dbl>     <dbl>     <dbl>   <dbl>      <dbl>    <dbl>
## 1 intercept 6805.     9929.        0.685   0.503 -14244.    27854.  
## 2 LastWeek     0.982     0.014    68.1     0          0.951     1.01
\end{verbatim}

\begin{Shaded}
\begin{Highlighting}[]
\NormalTok{regression\_points }\OtherTok{\textless{}{-}} \FunctionTok{get\_regression\_points}\NormalTok{(pbill\_model)}
\NormalTok{regression\_points}
\end{Highlighting}
\end{Shaded}

\begin{verbatim}
## # A tibble: 18 x 5
##       ID CurrentWeek LastWeek CurrentWeek_hat residual
##    <int>       <dbl>    <dbl>           <dbl>    <dbl>
##  1     1      684966   695437         689781.   -4815.
##  2     2      502367   498969         496833.    5534.
##  3     3      594474   598576         594655.    -181.
##  4     4      529298   528994         526320.    2978.
##  5     5      570254   562964         559681.   10573.
##  6     6      319959   282778         284516.   35443.
##  7     7      579126   583177         579532.    -406.
##  8     8      134042   152833         156899.  -22857.
##  9     9      105853   105698         110609.   -4756.
## 10    10      822775   836959         828767.   -5992.
## 11    11      949462   970190         959611.  -10149.
## 12    12      610007   651808         646933.  -36926.
## 13    13      386797   378238         378265.    8532.
## 14    14     1133034  1113510        1100362.   32672.
## 15    15      627609   614246         610045.   17564.
## 16    16      911727   933822         923894.  -12167.
## 17    17     1180266  1202536        1187793.   -7527.
## 18    18      479155   488624         486673.   -7518.
\end{verbatim}

\begin{enumerate}
\def\labelenumi{\arabic{enumi}.}
\item
  Find a 95\% confidence interval for the slope of the regression model,
  b1.

  Is 1 a plausible value for b1? Give a reason to support your answer.

  \hypertarget{standard-error-method}{%
  \paragraph{Standard Error Method}\label{standard-error-method}}
\item
  Test the null hypothesis 0 0 H : 10000 b = against a two-sided
  alternative. Interpret your result.
\item
  Use the fitted regression model to estimate the gross box office
  results for the current week (in \$) for a production with \$400,000
  in gross box office the previous week. Find a 95\% prediction interval
  for the gross box office results for the current week (in \$) for a
  production with \$400,000 in gross box office the previous week. Is
  \$450,000 a feasible value for the gross box office results in the
  current week, for a production with \$400,000 in gross box office the
  previous week? Give a reason to support your answer.
\item
  Some promoters of Broadway plays use the prediction rule that next
  week's gross box office results will be equal to this week's gross box
  office results. Comment on the appropriateness of this rule.
\end{enumerate}

\hypertarget{section-1}{%
\paragraph{2.2}\label{section-1}}

\hypertarget{lmr}{%
\subsection{LMR}\label{lmr}}

\hypertarget{section-2}{%
\paragraph{2.4}\label{section-2}}

You can also embed plots, for example:

\includegraphics{HW2_files/figure-latex/pressure-1.pdf}

Note that the \texttt{echo\ =\ FALSE} parameter was added to the code
chunk to prevent printing of the R code that generated the plot.

\hypertarget{section-3}{%
\paragraph{2.5}\label{section-3}}

\end{document}
